\section{Non‑Associative, Lie, and Quantum Algebras}\label{sec:algebras}

\subsection{Basic Definitions}

\begin{itemize}
    \item A \emph{non‑associative algebra} over a field \(k\) is a \(k\)-vector space \(A\) equipped with a bilinear multiplication \(A\times A\!\to\!A\) that is \emph{not} required to satisfy \((ab)c = a(bc)\).
    \item The \emph{associator} is \([a,b,c]=(ab)c-a(bc)\).  
          Associative algebras are those for which \([a,b,c]=0\) identically.
    \item Important classes include Lie, Jordan, alternative, and Malcev algebras, each characterised by specific identities satisfied by their associator or commutator.
\end{itemize}

\subsection{Lie Algebras}

\paragraph{Definition.}
A \emph{Lie algebra} \((\mathfrak{g},[\,,\,])\) over \(k\) is a vector space with a bilinear map \( [x,y] \) (the \emph{bracket}) satisfying
\[
[x,x]=0,\qquad
[x,[y,z]]+[y,[z,x]]+[z,[x,y]]=0.
\]

\paragraph{Structure Theory.}
\begin{itemize}
    \item \textbf{Ideals, solvable, nilpotent.}  
          A chain of derived algebras \(\mathfrak{g}^{(n)}=[\mathfrak{g}^{(n-1)},\mathfrak{g}^{(n-1)}]\) terminates in \(\{0\}\) precisely when \(\mathfrak{g}\) is solvable.
    \item \textbf{Semisimple Lie algebras.}  
          Over \(k=\mathbb{C}\), classification is via Dynkin diagrams \(\{A_n,B_n,C_n,D_n,E_6,E_7,E_8,F_4,G_2\}\).
    \item \textbf{Killing form.}  
          \(\kappa(x,y)=\operatorname{tr}(\operatorname{ad}x\,\operatorname{ad}y)\); \(\mathfrak{g}\) is semisimple iff \(\kappa\) is non‑degenerate.
\end{itemize}

\paragraph{Representation Theory.}
Finite‑dimensional modules are completely reducible for semisimple \(\mathfrak{g}\) (Weyl’s Theorem);  
irreducibles correspond to dominant integral weights.

\subsection{Jordan and Alternative Algebras}

\begin{itemize}
    \item A \emph{Jordan algebra} satisfies \(a\circ b = b\circ a\) and the Jordan identity  
          \((a^2\circ b)\circ a = a^2\circ (b\circ a)\).  
          Example: symmetric elements of an associative algebra with \(a\circ b=\tfrac12(ab+ba)\).
    \item An \emph{alternative algebra} fulfils the identities  
          \( (aa)b = a(ab) \) and \( a(bb) = (ab)b \).  
          The octonions \(\mathbb{O}\) are the archetypal non‑associative, alternative division algebra.
\end{itemize}

\subsection{Hopf Algebras}

A \emph{Hopf algebra} \(H\) over \(k\) is simultaneously an associative algebra \((H,m,1)\) and a co‑associative coalgebra \((H,\Delta,\varepsilon)\) equipped with an antipode \(S:H\!\to\!H\) satisfying
\[
m\!\circ\!(S\otimes\operatorname{id})\!\circ\!\Delta
\;=\;
m\!\circ\!(\operatorname{id}\otimes S)\!\circ\!\Delta
\;=\;
\eta\!\circ\!\varepsilon.
\]
Examples include group algebras \(kG\), coordinate algebras \(k[G]\) of affine algebraic groups, and universal enveloping algebras \(U(\mathfrak{g})\).

\subsection{Quantum Groups}

\begin{itemize}
    \item For a semisimple \(\mathfrak{g}\) and parameter \(q\in k^{\times}\), the \emph{quantum enveloping algebra} \(U_q(\mathfrak{g})\) deforms \(U(\mathfrak{g})\) while retaining a Hopf structure.
    \item Finite‑dimensional \(U_q(\mathfrak{g})\)-modules for generic \(q\) mirror the classical representation theory; at roots of unity new phenomena (e.g.\ tilting modules, modular categories) arise.
    \item The \emph{Yang–Baxter equation}, \(R_{12}R_{13}R_{23}=R_{23}R_{13}R_{12}\), underpins quantum groups and integrable models; \(R\)-matrices furnish solutions and give braided tensor categories.
\end{itemize}

\subsection{Illustrative Examples}

\begin{enumerate}
    \item \textbf{Heisenberg Lie Algebra.}  
          \(\mathfrak{h}_{2n+1}\) spanned by \(\{x_i,y_i,z\}\) with \([x_i,y_j]=\delta_{ij}z\) and \(z\) central.
    \item \textbf{Octonions.}  
          The 8‑dimensional algebra \(\mathbb{O}\) is alternative but non‑associative; its multiplicative unit sphere realises the exceptional Lie group \(G_2\) as automorphisms.
    \item \textbf{Classical Quantum Group.}  
          \(U_q(\mathfrak{sl}_2)\) generated by \(E,F,K,K^{-1}\) with relations  
          \(KEK^{-1}=q^{2}E\), \(KFK^{-1}=q^{-2}F\), \([E,F]=\tfrac{K-K^{-1}}{q-q^{-1}}\).
    \item \textbf{Hopf–Galois Extensions.}  
          For a finite group \(G\), a \(G\)-graded algebra \(A=\bigoplus_{g\in G}A_g\) with \(A_e = B\) gives a Hopf–Galois extension \(A/B\).
\end{enumerate}
