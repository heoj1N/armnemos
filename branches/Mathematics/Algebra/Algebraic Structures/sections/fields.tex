\section{Field and Galois Theory}

\subsection{Basic Definitions}

\paragraph{Fields.}
A \emph{field} \(F\) is a commutative ring with unity \((F,+,\cdot,1)\) in which every non‑zero element admits a multiplicative inverse.  
The \emph{characteristic} of \(F\) is the unique non‑negative integer \(\mathrm{char}(F)\) such that
\[
\underbrace{1 + 1 + \cdots + 1}_{n\ \text{times}} \;=\; 0 
\quad\Longleftrightarrow\quad n = \mathrm{char}(F).
\]
If no such \(n>0\) exists, we set \(\mathrm{char}(F)=0\).

\vspace{0.5em}
\noindent
Typical examples are the prime fields \(\mathbb{Q}\) and \(\mathbb{F}_p = \mathbb{Z}/p\mathbb{Z}\) for prime \(p\).

\subsection{Field Extensions}

\begin{itemize}
    \item An \emph{extension} \(E/F\) is a pair of fields with \(F\subseteq E\).  
          We regard \(E\) as a vector space over \(F\) and write \([E:F]\in\mathbb{N}\cup\{\infty\}\) for its dimension, called the \emph{degree} of the extension.
    \item (\textbf{Tower Law}) If \(K/E/F\) is a tower of fields, then
          \[
          [K:F] \;=\; [K:E]\,[E:F].
          \]
    \item An element \(\alpha\in E\) is \emph{algebraic} over \(F\) if it satisfies a non‑zero polynomial \(f\in F[x]\).  
          Otherwise \(\alpha\) is \emph{transcendental}.
\end{itemize}

\subsection{Splitting Fields and Algebraic Closures}

Given \(f\in F[x]\), a \emph{splitting field} \(L/F\) of \(f\) is an extension in which \(f\) factors completely into linear terms and \(L\) is generated by the roots of \(f\).  
Splitting fields exist and are unique up to \(F\)-isomorphism.  
An \emph{algebraic closure} \(\overline{F}\) of \(F\) is an algebraic extension in which every polynomial in \(F[x]\) splits; it is unique up to isomorphism.

\subsection{Separable and Inseparable Extensions}

Let \(E/F\) be algebraic.  An element \(\alpha\in E\) is \emph{separable} over \(F\) if its minimal polynomial \(m_\alpha(x)\) has distinct roots in a splitting field; otherwise \(\alpha\) is \emph{inseparable}.  
An extension is \emph{separable} if every element is separable.  
When \(\mathrm{char}(F)=0\) or \(F\) is finite, every algebraic extension is separable.

\subsection{Normal and Galois Extensions}

\begin{itemize}
    \item \(E/F\) is \emph{normal} if every \(F\)-embedding of \(E\) into an algebraic closure \(\overline{F}\) maps \(E\) onto itself.  Equivalently, \(E\) is a splitting field of a family of polynomials in \(F[x]\).
    \item \(E/F\) is \emph{Galois} if it is both normal and separable.  
          Its \emph{Galois group} is \(\mathrm{Gal}(E/F)=\{\,\sigma\in\mathrm{Aut}(E)\mid \sigma|_F=\mathrm{id}_F\,\}\).
\end{itemize}

\subsection{Fundamental Theorem of Galois Theory}

If \(E/F\) is a finite Galois extension, there is a one‑to‑one inclusion‑reversing correspondence
\[
\bigl\{\text{subgroups } H\le \mathrm{Gal}(E/F)\bigr\}
\;\longleftrightarrow\;
\bigl\{\text{intermediate fields } F\subseteq K\subseteq E\bigr\},
\]
given by \(H\mapsto E^{H}\) (fixed field) and \(K\mapsto \mathrm{Gal}(E/K)\).
Moreover, \(K/F\) is Galois \(\iff\) \(H\!\unlhd\!\mathrm{Gal}(E/F)\), in which case
\(\mathrm{Gal}(K/F)\cong \mathrm{Gal}(E/F)\!/H\).

\subsection{Finite (Galois) Fields}

For each prime power \(q=p^{n}\) there exists, up to isomorphism, a unique finite field \(\mathbb{F}_{q}\) of order \(q\).  
Its multiplicative group \(\mathbb{F}_{q}^{\times}\) is cyclic of order \(q-1\).  
The extension \(\mathbb{F}_{q}/\mathbb{F}_{p}\) is Galois with
\[
\mathrm{Gal}\bigl(\mathbb{F}_{q}/\mathbb{F}_{p}\bigr)
\;=\;
\langle\,\varphi:\,x\mapsto x^{p}\,\rangle
\cong
\mathbb{Z}/n\mathbb{Z},
\]
generated by the Frobenius automorphism.

\subsection{Solvability by Radicals}

A polynomial \(f\in F[x]\) is \emph{solvable by radicals} over \(F\) if its roots can be expressed using finitely many additions, multiplications, divisions and radical extractions starting from elements of \(F\).  
\[
\textit{Kronecker–Weber/Ruffini–Abel Criterion:}\quad
f \text{ is solvable by radicals }
\Longleftrightarrow
\mathrm{Gal}(L/F) \text{ is a solvable group},
\]
where \(L\) is the splitting field of \(f\).  
Since the symmetric group \(S_n\,(n\ge 5)\) is not solvable, a \emph{generic} quintic is not solvable by radicals.

\subsection{Illustrative Examples}

\begin{enumerate}
    \item \textbf{Quadratic Extensions.}  
          For a field \(F\) with \(\mathrm{char}(F)\neq 2\) and \(d\in F^{\times}\setminus (F^{\times})^{2}\), the extension \(F(\sqrt{d})/F\) is Galois with group of order \(2\).
    \item \textbf{Cyclotomic Fields.}  
          Let \(\zeta_n=\exp(2\pi i/n)\).  
          The extension \(\mathbb{Q}(\zeta_n)/\mathbb{Q}\) is Galois, and
          \(\mathrm{Gal}(\mathbb{Q}(\zeta_n)/\mathbb{Q}) \cong (\mathbb{Z}/n\mathbb{Z})^{\times}\).
    \item \textbf{Finite Fields.}  
          \(\mathbb{F}_{8} = \mathbb{F}_{2}[x]/(x^{3}+x+1)\) is a splitting field of \(x^{8}-x\) over \(\mathbb{F}_{2}\); its Galois group is cyclic of order \(3\).
    \item \textbf{Insolvable Quintic.}  
          The polynomial \(x^{5}-x-1\) has Galois group \(S_{5}\) over \(\mathbb{Q}\); hence it is not solvable by radicals.
\end{enumerate}
