\section{Ring, Module, and Algebra Theory}

\subsection{Basic Definitions}

\paragraph{Rings.}
A \emph{ring} \(R=(R,+,\cdot,1)\) is an abelian group under addition together with an associative
multiplication possessing a unity~\(1\neq 0\).
Unless specified otherwise, all rings are assumed to be unital and
multiplicatively commutative with~\(1\).

\paragraph{Ring Homomorphisms.}
A map \(\varphi:R\!\to\!S\) is a \emph{ring homomorphism} if
\(\varphi(1)=1\) and for all \(a,b\in R\)
\[
\varphi(a+b)=\varphi(a)+\varphi(b),\qquad
\varphi(ab)=\varphi(a)\,\varphi(b).
\]

\paragraph{Ideals and Quotients.}
A subset \(I\subseteq R\) is a (two‑sided) \emph{ideal} if
\(RI\subseteq I\) and \(IR\subseteq I\).
The quotient \(R/I\) inherits a natural ring structure;
the canonical projection \(\pi:R\!\twoheadrightarrow\!R/I\) is surjective with
\(\ker\pi = I\).

\medskip
Every surjective ring homomorphism \(\varphi:R\!\twoheadrightarrow\!S\) induces
\(R/\ker\varphi\;\cong\;S\) (\emph{First Isomorphism Theorem}).

\subsection{Ideals: Prime, Maximal, Radicals}

\begin{itemize}
    \item \textbf{Prime ideal.} \(P\!\subseteq\!R\) is prime if \(ab\in P\Rightarrow a\in P\ \text{or}\ b\in P\).  
          Equivalently \(R/P\) is an integral domain.
    \item \textbf{Maximal ideal.} \(M\!\subseteq\!R\) is maximal if no ideal lies properly between \(M\) and~\(R\); equivalently \(R/M\) is a field.
    \item \textbf{Jacobson radical.}  
          \(\operatorname{Jac}(R)=\bigcap_{M\text{ maximal}} M.\)
    \item \textbf{Nilradical.}
          \(\operatorname{Nil}(R)=\{\,a\in R\mid a^{n}=0 \text{ for some }n\ge1\}\)
          equals the intersection of all prime ideals.
    \item \textbf{Chinese Remainder Theorem.}
          If \(I_1,\ldots,I_k\lhd R\) are pairwise coprime, then
          \[
          R/\bigl(\cap_{i}I_i\bigr)\;\cong\;
          \prod_{i=1}^{k} R/I_i.
          \]
\end{itemize}

\subsection{Modules and Exact Sequences}

\paragraph{Modules.}
A (left) \emph{\(R\)-module} \(M\) is an abelian group
with scalar multiplication \(R\times M\!\to\!M\)
compatible with ring operations.  
Morphisms are \(R\)-linear maps.

\medskip\noindent
Submodules, quotients, and direct sums are defined analogously to vector spaces.
An exact sequence
\(0\!\to\!A\xrightarrow{f}B\xrightarrow{g}C\!\to\!0\)
means \(\ker g=\operatorname{im}f\) and \(f\) is injective, \(g\) surjective.

\paragraph{Projective, Injective, Flat.}
\begin{itemize}
    \item \(P\) is \emph{projective} if every surjection \(N\twoheadrightarrow M\) and map \(P\to M\)
          lift to \(P\to N\).
          Equivalently, \(P\) is a direct summand of a free module.
    \item \(I\) is \emph{injective} if every injection \(M\hookrightarrow N\)
          extends any map \(M\to I\) to \(N\to I\).
    \item \(F\) is \emph{flat} if the functor
          \(F\otimes_R -\) preserves exactness.
\end{itemize}

\paragraph{Homological Algebra.}
Given a projective resolution \(P_\bullet\to M\), the \emph{derived functors}
\[
\operatorname{Tor}^{R}_n(M,N),\qquad
\operatorname{Ext}^{n}_{R}(M,N)
\]
measure the failure of tensor and \(\operatorname{Hom}\) to be exact.

\subsection{Commutative Algebra Highlights}

\begin{itemize}
    \item \textbf{Noetherian rings.}  
          \(R\) is Noetherian if every ascending chain of ideals stabilises.
          Equivalent condition: every ideal is finitely generated.
    \item \textbf{Hilbert Basis Theorem.}  
          If \(R\) is Noetherian, so is \(R[x]\).
    \item \textbf{Primary Decomposition.}  
          In Noetherian rings every ideal can be written
          \(I = Q_1\cap\cdots\cap Q_r\) with each \(Q_i\) primary.
    \item \textbf{Localization.}  
          For \(S\subset R\) multiplicatively closed, \(S^{-1}R\)
          inverts elements of \(S\) and preserves many structural properties.
    \item \textbf{Krull Dimension.}  
          \(\dim R = \sup\{\ell\mid P_0\subsetneq\cdots\subsetneq P_\ell\text{ prime}\}\).
\end{itemize}

\subsection{Representation Theory of Algebras}

Let \(A\) be a (not‑necessarily‑commutative) finite‑dimensional \(k\)-algebra.

\begin{itemize}
    \item A left \(A\)-module is a \emph{representation} of \(A\).
          Simple modules correspond to
          primitive idempotents in \(A\!/\,\operatorname{Jac}(A)\).
    \item \(A\) is \emph{semisimple}  
          \(\Longleftrightarrow\) \(\operatorname{Jac}(A)=0\)
          \(\Longleftrightarrow\) every \(A\)-module is completely reducible  
          (Wedderburn–Artin).
    \item For a finite group \(G\) and field \(k\), the group algebra
          \(kG\) is semisimple iff \(\operatorname{char}k\nmid |G|\)
          (Maschke’s Theorem).
    \item The category \(\mathrm{Rep}_k(A)\) is abelian; 
          \(\operatorname{Hom}_A\) and \(\operatorname{Ext}^n_A\) control
          morphisms and extensions of representations.
\end{itemize}

\subsection{Illustrative Examples}

\begin{enumerate}
    \item \textbf{Principal Ideal Domains.}
          In a PID every submodule of a free module is free;  
          the classification of finitely generated modules yields
          the structure theorem for finitely generated abelian groups.
    \item \textbf{Polynomial Rings.}
          For \(R\) Noetherian, \(R[x_1,\dots,x_n]\) is Noetherian
          (Hilbert Basis), enabling Gröbner‑basis algorithms.
    \item \textbf{Quiver Algebras.}
          Path algebra \(kQ\) for a finite quiver \(Q\) encodes
          representations combinatorially;  
          Gabriel’s Theorem classifies quivers of finite representation type.
    \item \textbf{Artinian Rings.}
          Every left Artinian ring is Noetherian and
          decomposes into
          \(R\cong\prod_i M_{n_i}(D_i)\) with division rings \(D_i\).
\end{enumerate}
